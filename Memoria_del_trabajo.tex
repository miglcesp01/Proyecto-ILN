\documentclass[12pt,a4paper]{report}

% Paquetes necesarios
\usepackage[spanish]{babel}
\usepackage[utf8]{inputenc}
\usepackage{amsmath}
\usepackage{amsfonts}
\usepackage{amssymb}
\usepackage{graphicx}
\usepackage{hyperref}
\usepackage{url}
\usepackage{booktabs}
\usepackage{listings}
\usepackage{xcolor}
\usepackage{geometry}
\usepackage{titlesec}

% Configuración de márgenes
\geometry{margin=2.5cm}

% Configuración de listings para código
\definecolor{codegreen}{rgb}{0,0.6,0}
\definecolor{codegray}{rgb}{0.5,0.5,0.5}
\definecolor{codepurple}{rgb}{0.58,0,0.82}
\definecolor{backcolour}{rgb}{0.95,0.95,0.92}

\lstdefinestyle{mystyle}{
    backgroundcolor=\color{backcolour},
    commentstyle=\color{codegreen},
    keywordstyle=\color{magenta},
    numberstyle=\tiny\color{codegray},
    stringstyle=\color{codepurple},
    basicstyle=\ttfamily\footnotesize,
    breakatwhitespace=false,
    breaklines=true,
    captionpos=b,
    keepspaces=true,
    numbers=left,
    numbersep=5pt,
    showspaces=false,
    showstringspaces=false,
    showtabs=false,
    tabsize=2
}

\lstset{style=mystyle}

% Configuración de hiperenlaces
\hypersetup{
    colorlinks=true,
    linkcolor=blue,
    filecolor=magenta,
    urlcolor=cyan,
}

% Título del documento
\title{\Huge \textbf{Clasificación de Emociones con BERT}}
\author{
    \Large Miguel Angel Licea Céspedes \\
    \normalsize Universidad Politécnica de Valencia \\
    \normalsize malicces@upv.edu.es
    \and
    \Large Julié Arianne Pérez Vive  \\
    \normalsize Universidad Politécnica de Valencia \\
    \normalsize japerez@upv.edu.es
}
\date{\today}

\begin{document}

\maketitle
\tableofcontents
\newpage

\chapter*{Resumen}
\addcontentsline{toc}{chapter}{Resumen}

El presente trabajo aborda el desafío de la detección y clasificación automática de emociones en textos mediante el uso de técnicas avanzadas de Procesamiento del Lenguaje Natural (PLN). Se ha desarrollado un sistema basado en el modelo BERT (Bidirectional Encoder Representations from Transformers) que permite identificar 28 emociones distintas en textos, así como agruparlas en las categorías emocionales básicas de Ekman.

La investigación parte del conjunto de datos GoEmotions, que consiste en comentarios de Reddit anotados con múltiples emociones. Se implementó un modelo de clasificación multi-etiqueta utilizando BERT, que fue entrenado durante tres épocas con técnicas de preprocesamiento lingüístico como la lematización para mejorar la calidad de las predicciones. Los resultados muestran un rendimiento destacable en la identificación de emociones como ``gratitud'' (F1=0.908), ``amor'' (F1=0.764) y ``admiraci\'on'' (F1=0.742), mientras que otras emociones como ``duelo'' y ``nerviosismo'' resultaron más difíciles de detectar.

El sistema desarrollado se ha implementado además como una aplicación web interactiva utilizando Streamlit, permitiendo a los usuarios analizar textos en tiempo real y visualizar tanto las emociones detectadas como su distribución en categorías de Ekman. Esta herramienta tiene potenciales aplicaciones en análisis de sentimientos en redes sociales, servicio al cliente, psicología computacional y estudios de comportamiento humano en entornos digitales.

La metodología aplicada y los resultados obtenidos demuestran el potencial de los modelos de lenguaje pre-entrenados para tareas complejas de clasificación emocional, representando un avance significativo respecto a técnicas tradicionales basadas en léxicos o enfoques estadísticos simples.

\chapter{Descripción del Trabajo}

\section{Motivación y Objetivos}

La comprensión de las emociones humanas expresadas en textos representa uno de los mayores desafíos del Procesamiento del Lenguaje Natural actual. A diferencia del análisis de sentimientos tradicional, que suele limitarse a clasificaciones positivas, negativas o neutras, la detección de emociones específicas requiere modelos con una comprensión más profunda del lenguaje y sus matices.

La principal motivación de este trabajo surge de la necesidad de desarrollar sistemas capaces de identificar un espectro más amplio de emociones en textos, lo que permitiría aplicaciones más sofisticadas en diversos campos como:

\begin{itemize}
  \item Análisis de la percepción de usuarios en redes sociales
  \item Mejora de sistemas de atención al cliente
  \item Apoyo a estudios psicológicos y sociológicos
  \item Desarrollo de agentes conversacionales más empáticos
  \item Monitorización de bienestar emocional en entornos digitales
\end{itemize}

El conjunto de datos GoEmotions, publicado por Google Research, ofrece una oportunidad única para abordar este problema, ya que proporciona un corpus de comentarios de Reddit anotados con 28 emociones diferentes, permitiendo entrenar modelos de mayor granularidad emocional.

Los objetivos específicos planteados para este proyecto fueron:

\begin{enumerate}
  \item Implementar un clasificador multi-etiqueta basado en BERT capaz de detectar 28 categorías emocionales distintas en textos.
  \item Explorar técnicas de preprocesamiento lingüístico (como la lematización) para mejorar el rendimiento del modelo.
  \item Desarrollar una metodología para agrupar las emociones detectadas en las seis categorías básicas de Ekman (alegría, tristeza, ira, miedo, sorpresa y asco) más la categoría neutral.
  \item Crear una aplicación web interactiva que permita a los usuarios analizar textos y visualizar las emociones detectadas.
  \item Evaluar el rendimiento del modelo mediante métricas estándar como F1-score, identificando fortalezas y debilidades en la detección de cada tipo de emoción.
\end{enumerate}

Este trabajo se alinea con la tendencia actual en PLN de ir más allá del análisis de sentimientos binario hacia una comprensión más matizada de las expresiones emocionales humanas, lo que representa un paso importante hacia sistemas computacionales con mayor inteligencia emocional.

\section{Herramientas Utilizadas}

Para el desarrollo de este proyecto de clasificación de emociones se utilizaron diversas tecnologías y bibliotecas especializadas en aprendizaje profundo y procesamiento del lenguaje natural:

\subsection{Frameworks y Bibliotecas Principales}

\begin{itemize}
  \item \textbf{PyTorch (1.x)}: Framework de aprendizaje profundo que proporciona la infraestructura para implementar y entrenar el modelo neuronal. Se eligió por su flexibilidad y por ofrecer un buen equilibrio entre rendimiento y facilidad de uso para la experimentación.

  \item \textbf{Transformers (Hugging Face)}: Biblioteca que proporciona implementaciones estado del arte de modelos de lenguaje como BERT. Se utilizó para acceder al modelo BERT pre-entrenado y para los tokenizadores especializados.

  \item \textbf{BERT (Bidirectional Encoder Representations from Transformers)}: Modelo de lenguaje pre-entrenado desarrollado por Google que captura el contexto bidireccional de las palabras. Se utilizó la versión base sin capa superior (\texttt{bert-base-uncased}) como punto de partida para nuestra tarea de clasificación.

  \item \textbf{NLTK (Natural Language Toolkit)}: Biblioteca para procesamiento de lenguaje natural utilizada específicamente para tareas de tokenización y lematización en el preprocesamiento textual.

  \item \textbf{Streamlit}: Framework para la creación de aplicaciones web interactivas en Python. Se empleó para desarrollar la interfaz de usuario que permite probar el clasificador de emociones en tiempo real.
\end{itemize}

\subsection{Bibliotecas de Análisis y Visualización}

\begin{itemize}
  \item \textbf{Pandas}: Utilizada para el procesamiento y manipulación eficiente de los datos tabulares, especialmente para cargar y transformar el conjunto de datos GoEmotions.

  \item \textbf{NumPy}: Empleada para operaciones matemáticas y manipulación de arrays multidimensionales, esencial para el procesamiento de las representaciones vectoriales.

  \item \textbf{Matplotlib y Pyplot}: Usadas para la generación de gráficos y visualizaciones que muestran las distribuciones de emociones y los resultados de las predicciones.

  \item \textbf{Scikit-learn}: Proporciona herramientas para la evaluación del modelo (métricas como F1-score) y para el procesamiento de etiquetas múltiples (MultiLabelBinarizer).
\end{itemize}

\subsection{Herramientas de Desarrollo}

\begin{itemize}
  \item \textbf{Visual Studio Code}: IDE principal utilizado para el desarrollo del código.

  \item \textbf{Git/GitHub}: Para control de versiones y almacenamiento del repositorio del proyecto.
\end{itemize}

\subsection{Datos y Recursos}

\begin{itemize}
  \item \textbf{GoEmotions Dataset}: Conjunto de datos de Google Research que contiene aproximadamente 58,000 comentarios de Reddit anotados con 28 emociones diferentes. Este recurso fue fundamental para entrenar el modelo, ya que proporciona datos etiquetados de alta calidad.

  \item \textbf{Mapeo de Ekman}: Se desarrolló un mapeo personalizado para agrupar las 28 emociones del dataset GoEmotions en las seis categorías emocionales fundamentales definidas por Paul Ekman (alegría, tristeza, ira, miedo, sorpresa, asco) más la categoría neutral.
\end{itemize}

La combinación de estas herramientas permitió desarrollar un sistema integral que abarca desde el preprocesamiento de datos, entrenamiento del modelo, evaluación del rendimiento, hasta la implementación de una interfaz de usuario para facilitar el uso del clasificador.

\section{Desarrollo del Trabajo}

El desarrollo del proyecto de clasificación de emociones con BERT siguió una metodología estructurada que abarcó desde la comprensión del problema hasta la implementación de una aplicación funcional. A continuación, se detallan las etapas principales:

\subsection{Comprensión y Exploración del Dataset GoEmotions}

El punto de partida fue el estudio detallado del conjunto de datos GoEmotions, que presenta las siguientes características:
\begin{itemize}
  \item Aproximadamente 58,000 comentarios de Reddit anotados manualmente
  \item Cada comentario puede estar asociado con múltiples emociones (clasificación multi-etiqueta)
  \item 28 categorías emocionales distintas, incluyendo una categoría ``neutral''
  \item División en conjuntos de entrenamiento, validación y prueba
\end{itemize}

Se realizó un análisis exploratorio para entender la distribución de las emociones en el dataset, identificando desbalances significativos. Por ejemplo, emociones como ``gratitud'', ``admiración'' y ``alegría'' están mucho más representadas que otras como ``duelo'', ``nerviosismo'' o ``vergüenza''. Este desbalance tendría impacto en el rendimiento del modelo.

\subsection{Diseño e Implementación del Modelo}

Para abordar la tarea de clasificación multi-etiqueta, se diseñó un modelo basado en BERT con las siguientes características:

\begin{lstlisting}[language=Python]
class BertForMultiLabelClassification(nn.Module):
    def __init__(self, num_labels):
        super(BertForMultiLabelClassification, self).__init__()
        self.bert = BertModel.from_pretrained('bert-base-uncased')
        self.dropout = nn.Dropout(0.1)
        self.classifier = nn.Linear(self.bert.config.hidden_size, num_labels)
        self.sigmoid = nn.Sigmoid()
\end{lstlisting}

La arquitectura consta de:
\begin{enumerate}
  \item Una capa BERT base pre-entrenada que genera representaciones contextuales
  \item Una capa de dropout (0.1) para regularización
  \item Una capa lineal que proyecta las representaciones a 28 dimensiones (una por emoción)
  \item Una función de activación sigmoid que transforma las salidas en probabilidades independientes
\end{enumerate}

Se utilizó la función de pérdida Binary Cross-Entropy (BCE), adecuada para problemas de clasificación multi-etiqueta:

\begin{lstlisting}[language=Python]
loss_fct = nn.BCELoss()
loss = loss_fct(output, labels)
\end{lstlisting}

\subsection{Preprocesamiento Lingüístico}

Se implementaron dos versiones del modelo:
\begin{itemize}
  \item \textbf{Básica}: Utilizando el texto original con la tokenización estándar de BERT
  \item \textbf{Avanzada}: Incorporando lematización para reducir la variabilidad lingüística
\end{itemize}

Para la lematización se utilizó el WordNet Lemmatizer de NLTK:

\begin{lstlisting}[language=Python]
def preprocess_text(text):
    # Tokenize
    tokens = word_tokenize(text.lower())
    # Lemmatize
    lemmatized_tokens = [lemmatizer.lemmatize(token) for token in tokens]
    # Join back to text
    processed_text = ' '.join(lemmatized_tokens)
    return processed_text
\end{lstlisting}

Esta técnica permitió normalizar palabras con la misma raíz semántica, como "corriendo", "correr", "corrió" a su forma base "correr", potencialmente mejorando la capacidad del modelo para generalizar.

\subsection{Entrenamiento del Modelo}

El proceso de entrenamiento se configuró con los siguientes parámetros:
\begin{itemize}
  \item Optimizador: AdamW con learning rate de 2e-5
  \item Batch size: 16
  \item Número de épocas: 3
  \item Dispositivo de entrenamiento: CPU/GPU dependiendo de la disponibilidad
\end{itemize}

El entrenamiento siguió este flujo:
\begin{enumerate}
  \item Carga de datos mediante la clase personalizada \texttt{GoEmotionsDataset}
  \item Tokenización y preparación de batches con DataLoader
  \item Forward pass a través del modelo
  \item Cálculo de pérdida y backpropagation
  \item Actualización de pesos
\end{enumerate}

Durante el entrenamiento, se monitorizó la pérdida en los conjuntos de entrenamiento y validación. El modelo mostró una convergencia progresiva con la pérdida de entrenamiento reduciéndose de aproximadamente 0.10 a 0.06 a lo largo de las épocas.

\subsection{Desarrollo del Sistema de Categorización de Ekman}

Adicionalmente al modelo de 28 emociones, se implementó un sistema para mapear estas emociones a las 6 categorías fundamentales de Ekman (más neutral):
\begin{itemize}
  \item Alegría (joy)
  \item Tristeza (sadness)
  \item Ira (anger)
  \item Miedo (fear)
  \item Sorpresa (surprise)
  \item Asco (disgust)
  \item Neutral
\end{itemize}

Este mapeo se implementó mediante un diccionario JSON que asigna cada emoción específica a una categoría de Ekman:

\begin{lstlisting}[language=JSON]
{
  "anger": ["anger", "annoyance", "disapproval"],
  "disgust": ["disgust"],
  "fear": ["fear", "nervousness"],
  "joy": ["joy", "amusement", "approval", "excitement", "gratitude", 
          "love", "optimism", "relief", "pride", "admiration", "desire", "caring"],
  "sadness": ["sadness", "disappointment", "embarrassment", "grief", 
              "remorse"],
  "surprise": ["surprise", "realization", "confusion", "curiosity"]
}
\end{lstlisting}

El sistema calcula la distribución de Ekman sumando las probabilidades de todas las emociones que pertenecen a cada categoría principal, proporcionando una visión más holística del estado emocional del texto.

\subsection{Evaluación del Rendimiento}

Para evaluar el modelo, se utilizó principalmente el F1-score por clase, adecuado para conjuntos de datos desequilibrados, además de la pérdida de validación. Los resultados mostraron variaciones significativas entre emociones:

\begin{itemize}
  \item Emociones con buen rendimiento (F1 $>$ 0.75):
  \begin{itemize}
    \item Gratitud (0.908)
    \item Amor (0.764)
    \item Admiración (0.742)
    \item Diversión (0.784)
  \end{itemize}
  
  \item Emociones con rendimiento medio (0.4 $<$ F1 $<$ 0.75):
  \begin{itemize}
    \item Alegría (0.468)
    \item Tristeza (0.496)
    \item Sorpresa (0.523)
    \item Neutral (0.647)
  \end{itemize}
  
  \item Emociones difíciles de detectar (F1 $<$ 0.2 ó 0):
  \begin{itemize}
    \item Duelo (0.000)
    \item Nerviosismo (0.000)
    \item Orgullo (0.000)
    \item Decepción (0.047)
  \end{itemize}
\end{itemize}

El modelo con lematización mostró mejoras en varias categorías emocionales en comparación con el modelo base, particularmente en emociones como "remordimiento" y "optimismo".

\subsection{Desarrollo de la Aplicación Web}

Finalmente, se implementó una aplicación web utilizando Streamlit para demostrar las capacidades del modelo. La aplicación incluye:

\begin{enumerate}
  \item \textbf{Interfaz de entrada de texto}: Donde los usuarios pueden escribir o pegar texto para analizar.
  \item \textbf{Ajuste de umbral}: Permite al usuario controlar la sensibilidad de la detección de emociones.
  \item \textbf{Visualizaciones}:
  \begin{itemize}
    \item Gráfico de barras horizontales mostrando las emociones detectadas y sus probabilidades
    \item Gráfico circular que representa la distribución de categorías de Ekman
    \item Tabla detallada con todas las emociones detectadas
  \end{itemize}
\end{enumerate}

La aplicación carga el modelo previamente entrenado y realiza las predicciones en tiempo real, ofreciendo una experiencia interactiva para explorar la capacidad del sistema en diferentes tipos de textos.

Durante el desarrollo se enfrentaron varios desafíos, entre ellos:
\begin{itemize}
  \item El manejo eficiente de la memoria durante el entrenamiento con datasets grandes
  \item La optimización del rendimiento para emociones con poca representación
  \item La correcta interpretación de la clasificación multi-etiqueta
  \item La integración del modelo en la aplicación web asegurando tiempos de respuesta rápidos
\end{itemize}

Estos desafíos se abordaron mediante ajustes en la arquitectura del modelo, técnicas de preprocesamiento lingüístico como la lematización, y la optimización del código de la aplicación.

\section{Resultados}

El desarrollo y evaluación del modelo de clasificación de emociones arrojó resultados significativos que permiten valorar su rendimiento y potencial aplicabilidad. A continuación, se presentan los hallazgos principales organizados por aspectos clave:

\subsection{Rendimiento General del Modelo}

El modelo final, entrenado durante 3 épocas con lematización, alcanzó los siguientes indicadores generales:
\begin{itemize}
  \item \textbf{Loss de entrenamiento final}: 0.0675
  \item \textbf{Loss de validación final}: 0.0841
  \item \textbf{F1-score macro promedio}: 0.3734
\end{itemize}

Este F1-score macro refleja el promedio de los F1-scores individuales de cada emoción, sin ponderación por frecuencia, lo que proporciona una visión más equilibrada del rendimiento en un dataset desbalanceado.

\subsection{Rendimiento por Categoría Emocional}

A continuación se muestran los F1-scores obtenidos para cada emoción, agrupados por su rendimiento:

\begin{table}[htbp]
  \centering
  \caption{Emociones con alto rendimiento (F1 $>$ 0.7)}
  \begin{tabular}{lc}
    \toprule
    \textbf{Emoción} & \textbf{F1-score} \\
    \midrule
    Gratitud (gratitude) & 0.9080 \\
    Diversión (amusement) & 0.7843 \\
    Amor (love) & 0.7637 \\
    Admiración (admiration) & 0.7418 \\
    \bottomrule
  \end{tabular}
\end{table}

\begin{table}[htbp]
  \centering
  \caption{Emociones con rendimiento aceptable (0.5 $<$ F1 $<$ 0.7)}
  \begin{tabular}{lc}
    \toprule
    \textbf{Emoción} & \textbf{F1-score} \\
    \midrule
    Neutral & 0.6471 \\
    Optimismo (optimism) & 0.6072 \\
    Miedo (fear) & 0.5778 \\
    Remordimiento (remorse) & 0.5586 \\
    Sorpresa (surprise) & 0.5229 \\
    \bottomrule
  \end{tabular}
\end{table}

\begin{table}[htbp]
  \centering
  \caption{Emociones con rendimiento medio (0.3 $<$ F1 $<$ 0.5)}
  \begin{tabular}{lc}
    \toprule
    \textbf{Emoción} & \textbf{F1-score} \\
    \midrule
    Tristeza (sadness) & 0.4959 \\
    Alegría (joy) & 0.4683 \\
    Ira (anger) & 0.4765 \\
    Disgusto (disgust) & 0.3969 \\
    Confusión (confusion) & 0.3364 \\
    \bottomrule
  \end{tabular}
\end{table}

\begin{table}[htbp]
  \centering
  \caption{Emociones con bajo rendimiento (F1 $<$ 0.3)}
  \begin{tabular}{lc}
    \toprule
    \textbf{Emoción} & \textbf{F1-score} \\
    \midrule
    Molestia (annoyance) & 0.2701 \\
    Entusiasmo (excitement) & 0.2459 \\
    Deseo (desire) & 0.2418 \\
    Aprobación (approval) & 0.2398 \\
    Curiosidad (curiosity) & 0.2290 \\
    Desaprobación (disapproval) & 0.1924 \\
    Vergüenza (embarrassment) & 0.1579 \\
    Realización (realization) & 0.0758 \\
    Decepción (disappointment) & 0.0473 \\
    Duelo (grief) & 0.0000 \\
    Nerviosismo (nervousness) & 0.0000 \\
    Orgullo (pride) & 0.0000 \\
    Alivio (relief) & 0.0000 \\
    \bottomrule
  \end{tabular}
\end{table}

La variabilidad en el rendimiento está directamente relacionada con:
\begin{enumerate}
  \item La frecuencia de cada emoción en el conjunto de entrenamiento
  \item La ambigüedad inherente de ciertas emociones (ej. la distinción entre ``annoyance'' y ``anger'')
  \item La complejidad de las expresiones textuales asociadas a cada emoción
\end{enumerate}

\subsection{Distribución por Categorías de Ekman}

Al agrupar las emociones en las categorías de Ekman, el modelo mostró un rendimiento más equilibrado. La representación consolidada mejoró la fiabilidad en casos donde emociones individuales eran difíciles de detectar. Por ejemplo, aunque ``nervousness'' obtuvo un F1 de 0, la categoría general ``fear'' alcanzó un F1 de 0.5778.

Las categorías de Ekman con mejor rendimiento fueron:
\begin{enumerate}
  \item Alegría (joy): 0.67 (promedio ponderado)
  \item Asco (disgust): 0.40
  \item Tristeza (sadness): 0.37
  \item Sorpresa (surprise): 0.33
\end{enumerate}

\subsection{Impacto de la Lematización}

La aplicación de lematización como técnica de preprocesamiento mostró un impacto positivo en el rendimiento global:

\begin{table}[htbp]
  \centering
  \caption{Impacto de la lematización en el rendimiento del modelo}
  \begin{tabular}{lccc}
    \toprule
    \textbf{Métrica} & \textbf{Sin lematización} & \textbf{Con lematización} & \textbf{Mejora} \\
    \midrule
    F1-score macro promedio & 0.2727 & 0.3734 & +36.9\% \\
    Loss de validación & 0.0935 & 0.0841 & +10.1\% \\
    \bottomrule
  \end{tabular}
\end{table}

El impacto fue particularmente notable en emociones con expresiones lingüísticas variadas como ``remorse'' (+21.3\%) y ``optimism'' (+5.2\%).

\subsection{Análisis de Errores}

Se identificaron principalmente tres tipos de errores:

\begin{enumerate}
  \item \textbf{Falsos negativos en emociones poco frecuentes}: Emociones como ``grief'', ``pride'' y ``nervousness'' rara vez fueron detectadas, incluso cuando estaban presentes.

  \item \textbf{Confusión entre emociones relacionadas}: El modelo a menudo confundió pares de emociones semánticamente cercanas:
  \begin{itemize}
    \item Admiración vs. Amor
    \item Ira vs. Molestia
    \item Sorpresa vs. Realización
  \end{itemize}

  \item \textbf{Detección excesiva de emociones dominantes}: Emociones como ``gratitude'' y ``admiration'' tendían a ser predichas con más frecuencia debido a su sobrerrepresentación en los datos de entrenamiento.
\end{enumerate}

\subsection{Resultados de la Aplicación Web}

La implementación de la aplicación web con Streamlit demostró:

\begin{enumerate}
  \item \textbf{Tiempos de respuesta adecuados}: Predicciones completadas en aproximadamente 0.3-0.5 segundos por texto en CPU estándar.

  \item \textbf{Consistencia en las predicciones}: Los resultados fueron consistentes con las métricas observadas durante la evaluación del modelo.

  \item \textbf{Visualizaciones informativas}: Las representaciones gráficas de emociones y categorías de Ekman proporcionaron una interpretación intuitiva de los resultados.
\end{enumerate}

La capacidad de ajustar el umbral de detección resultó particularmente útil, permitiendo a los usuarios controlar el equilibrio entre precisión y exhaustividad según sus necesidades específicas.

\subsection{Comparación con el Estado del Arte}

Aunque no era un objetivo principal del proyecto, se realizó una breve comparación con otros trabajos en el ámbito de la detección de emociones:

\begin{table}[htbp]
  \centering
  \caption{Comparación con otros enfoques}
  \begin{tabular}{lc}
    \toprule
    \textbf{Enfoque} & \textbf{F1-score macro en GoEmotions} \\
    \midrule
    Modelo desarrollado (BERT + lematización) & 0.37 \\
    BERT base (literatura) & 0.46 \\
    XLNet (literatura) & 0.50 \\
    Modelos basados en reglas/léxicos & 0.20-0.25 \\
    \bottomrule
  \end{tabular}
\end{table}

El modelo desarrollado mostró un rendimiento razonable considerando el alcance y recursos del proyecto, situándose por encima de los enfoques tradicionales basados en léxicos y a una distancia aceptable de implementaciones más avanzadas.

La implementación de un sistema completo de clasificación y visualización, junto con la categorización de Ekman, añade un valor significativo más allá del rendimiento bruto del clasificador.

\section{Conclusiones}

El desarrollo de este proyecto de clasificación de emociones con BERT ha permitido obtener conclusiones relevantes tanto desde el punto de vista técnico como aplicado:

\subsection{Viabilidad de la Clasificación Emocional Multi-etiqueta}

El modelo implementado demuestra que es posible abordar la clasificación multi-etiqueta de emociones con un nivel de precisión aceptable utilizando arquitecturas transformer pre-entrenadas. La capacidad de BERT para capturar contextos bidireccionales resulta fundamental en una tarea donde los matices lingüísticos son determinantes para distinguir entre estados emocionales cercanos.

Un hallazgo importante es la diferencia significativa en el rendimiento de detección según la emoción específica, lo que subraya la complejidad inherente de mapear lenguaje a estados emocionales. Algunas emociones presentan patrones lingüísticos más consistentes y reconocibles (gratitud, admiración, amor) mientras que otras (orgullo, nerviosismo, duelo) requieren un tratamiento más sofisticado.

\subsection{Impacto del Preprocesamiento Lingüístico}

Los resultados evidencian que técnicas de preprocesamiento relativamente simples como la lematización pueden tener un impacto significativo en el rendimiento de los modelos de clasificación emocional. La mejora global de casi un 37\% en F1-score macro al implementar lematización confirma la importancia de considerar la morfología de las palabras en la detección de emociones.

Esto sugiere que, a pesar del poder de los modelos pre-entrenados, el preprocesamiento específico para la tarea sigue siendo relevante y debe considerarse como parte integral del desarrollo de soluciones de PLN.

\subsection{Potencial de la Categorización de Ekman}

La agrupación de emociones en las categorías de Ekman ha demostrado ser un enfoque valioso para proporcionar una visión más estable y holística del estado emocional expresado en un texto. Esta abstracción a un nivel superior permite superar algunas de las limitaciones de la clasificación en emociones específicas, especialmente para aquellas con bajo rendimiento individual.

El modelo jerárquico (emociones específicas → categorías de Ekman) proporciona un balance entre granularidad y confiabilidad que puede ser ajustado según las necesidades de la aplicación.

\subsection{Desafíos del Desbalance de Datos}

El conjunto de datos GoEmotions, como muchos datasets reales, presenta un desbalance significativo en la distribución de clases. Los resultados confirman que las emociones con menor representación en los datos de entrenamiento tienden a tener un rendimiento predictivo inferior.

Este hallazgo resalta la importancia de considerar técnicas específicas para conjuntos de datos desbalanceados en futuros trabajos, como estrategias de sobremuestreo, submuestreo o ajustes en las funciones de pérdida.

\subsection{Aplicabilidad Práctica}

La implementación exitosa de una aplicación web demuestra la viabilidad de utilizar modelos de detección de emociones en entornos reales y con tiempos de respuesta aceptables. El sistema desarrollado podría aplicarse en diversos contextos como:

\begin{itemize}
  \item \textbf{Monitorización de feedback de usuarios}: Analizando comentarios o reseñas para comprender la carga emocional asociada.
  \item \textbf{Mejora de chatbots y asistentes virtuales}: Permitiendo respuestas más empáticas basadas en la emoción detectada.
  \item \textbf{Análisis de redes sociales}: Monitorizando tendencias emocionales en tiempo real sobre temas específicos.
  \item \textbf{Herramientas de apoyo para escritores}: Ayudando a evaluar el tono emocional de textos durante su creación.
  \item \textbf{Aplicaciones educativas}: Analizando la respuesta emocional de estudiantes en entornos de aprendizaje en línea.
\end{itemize}

\subsection{Limitaciones y Líneas Futuras de Investigación}

A pesar de los resultados prometedores, es importante reconocer las limitaciones del sistema:

\begin{enumerate}
  \item \textbf{Especificidad contextual}: El modelo está entrenado principalmente con comentarios de Reddit, lo que podría limitar su generalización a otros contextos lingüísticos o dominios.

  \item \textbf{Sesgo cultural}: La interpretación de emociones puede variar significativamente entre culturas, y el modelo refleja principalmente perspectivas occidentales sobre la expresión emocional.

  \item \textbf{Complejidad lingüística}: Aspectos como la ironía, el sarcasmo o expresiones idiomáticas siguen siendo desafíos importantes que el modelo actual no aborda completamente.
\end{enumerate}

Para futuras investigaciones, sería valioso explorar:

\begin{itemize}
  \item La incorporación de técnicas de data augmentation para mejorar el rendimiento en emociones poco representadas.
  \item La adaptación del modelo a dominios específicos mediante fine-tuning adicional.
  \item La exploración de arquitecturas más avanzadas como XLNet, RoBERTa o T5.
  \item La integración de señales multimodales (texto, audio, imagen) para una detección emocional más robusta.
  \item El desarrollo de versiones más eficientes del modelo para su uso en dispositivos con recursos limitados.
\end{itemize}

En conclusión, el proyecto ha demostrado tanto el potencial como los desafíos de la detección automática de emociones en textos, abriendo camino para aplicaciones más sofisticadas que puedan comprender y responder a los matices emocionales de la comunicación humana. La combinación de arquitecturas transformer con técnicas de procesamiento lingüístico específicas representa un enfoque prometedor para seguir avanzando en este campo.

\chapter{Valoración Personal}

El desarrollo de este proyecto de clasificación de emociones ha representado una experiencia enormemente enriquecedora y desafiante desde múltiples perspectivas. A nivel académico, me ha permitido profundizar en la comprensión y aplicación práctica de los modelos transformer, particularmente BERT, uno de los avances más significativos en el campo del procesamiento del lenguaje natural en los últimos años.

Uno de los aspectos más gratificantes ha sido observar cómo la teoría aprendida durante el curso se transformaba en aplicaciones concretas y funcionales. El proceso de adaptación de un modelo pre-entrenado para una tarea específica de clasificación multi-etiqueta ha sido especialmente instructivo, permitiéndome entender mejor las complejidades del fine-tuning y las consideraciones especiales que requieren las tareas de clasificación con múltiples categorías simultáneas.

Sin embargo, el proyecto no estuvo exento de desafíos significativos. Particularmente complicado resultó el manejo del desbalance de clases en el dataset GoEmotions, que refleja fielmente un problema común en aplicaciones del mundo real: algunas emociones son expresadas con mucha más frecuencia que otras en el lenguaje cotidiano. Encontrar estrategias para mejorar el rendimiento en emociones poco representadas, sin sacrificar la precisión en las más comunes, fue un ejercicio valioso que me obligó a investigar y experimentar con diversas técnicas.

La implementación de la lematización como estrategia de preprocesamiento y su impacto positivo en los resultados del modelo fue una de las satisfacciones más grandes del proyecto. Este hallazgo reforzó mi convicción de que, incluso en la era de los modelos pre-entrenados, el conocimiento lingüístico específico sigue siendo relevante y puede marcar diferencias significativas en el rendimiento.

Desde el punto de vista técnico, el desarrollo de la aplicación web con Streamlit representó una oportunidad para ampliar mis habilidades más allá del modelado puro, abordando aspectos de visualización de datos y desarrollo de interfaces que son cruciales para hacer que las soluciones de inteligencia artificial sean accesibles y útiles para usuarios finales.

Una de las reflexiones más importantes que me llevo de este proyecto es la comprensión de las limitaciones actuales de los sistemas de PLN para capturar toda la riqueza y complejidad de las emociones humanas. El hecho de que emociones como el duelo, el nerviosismo o el orgullo resultaran tan difíciles de detectar automáticamente nos recuerda cuánto camino queda por recorrer en este campo.

A nivel personal, este proyecto ha reforzado mi interés en la intersección entre el procesamiento del lenguaje natural y la psicología computacional. Creo firmemente que el desarrollo de sistemas capaces de comprender y responder apropiadamente a las emociones humanas será un área de investigación cada vez más relevante, con aplicaciones significativas en ámbitos como la salud mental, la educación y las interacciones humano-máquina.

En resumen, este trabajo no solo me ha permitido aplicar y expandir mis conocimientos técnicos, sino que también ha abierto mi perspectiva sobre las posibilidades y desafíos futuros en el campo del análisis emocional automatizado. Me siento satisfecho con los resultados obtenidos, consciente de las limitaciones del enfoque actual, y motivado para seguir explorando y contribuyendo a este fascinante campo de investigación.

\chapter{Enlace al Código Implementado}

El código fuente completo de este proyecto está disponible en el siguiente repositorio de GitHub:

\url{https://github.com/miglcesp01/Proyecto-ILN}

\section{Estructura del Repositorio}

El repositorio está organizado de la siguiente manera:

\begin{lstlisting}
emotion-classifier-bert/
|-- app.py                      # Streamlit web application
|-- emotion_classifier.py       # Main classifier implementation
|-- data/                       # Training and evaluation data
|   |-- emotions.txt            # List of emotions
|   |-- ekman_mapping.json      # Mapping of emotions to Ekman categories 
|   |-- train.tsv               # Training data
|   |-- dev.tsv                 # Validation data
|   |-- test.tsv                # Test data
|-- training/                   # Trained models
    |-- goemotions_bert_model_lemma.pt        # Model with lemmatization
    |-- goemotions_bert_model_dev_lemma.pt    # Development version
\end{lstlisting}

\section{Instrucciones de Ejecución}

Para ejecutar la aplicación web de clasificación de emociones:

\begin{enumerate}
  \item Clonar el repositorio:
  \begin{lstlisting}[language=bash]
git clone https://github.com/username/emotion-classifier-bert.git
cd emotion-classifier-bert\end{lstlisting}

  \item Instalar las dependencias:
  \begin{lstlisting}[language=bash]
pip install -r requirements.txt\end{lstlisting}

  \item Ejecutar la aplicación web:
  \begin{lstlisting}[language=bash]
streamlit run app.py\end{lstlisting}

  \item Acceder a la aplicación a través del navegador web en la dirección indicada (generalmente http://localhost:8501).
\end{enumerate}

Para reentrenar el modelo o experimentar con diferentes configuraciones, se puede ejecutar el script principal:

\begin{lstlisting}[language=bash]
python emotion_classifier.py
\end{lstlisting}

El código está documentado con comentarios detallados para facilitar su comprensión y modificación.


\begin{thebibliography}{7}

\bibitem{devlin2019} Devlin, Jacob; Chang, Ming-Wei; Lee, Kenton; and Toutanova, Kristina. 2019. BERT: Pre-training of Deep Bidirectional Transformers for Language Understanding. In Proceedings of NAACL-HLT 2019. \url{https://arxiv.org/abs/1810.04805}.

\bibitem{demszky2020} Demszky, Dorottya; Movshovitz-Attias, Dana; Ko, Jeongwoo; Cowen, Alan; Nemade, Gaurav; and Ravi, Sujith. 2020. GoEmotions: A Dataset of Fine-Grained Emotions. In Proceedings of the 58th Annual Meeting of the Association for Computational Linguistics (ACL 2020). \url{https://aclanthology.org/2020.acl-main.372}.

\bibitem{zhang2014} Zhang, Min-Ling; and Zhou, Zhi-Hua. 2014. A Review on Multi-Label Learning Algorithms. IEEE Transactions on Knowledge and Data Engineering, 26(8):1819–1837. \url{https://doi.org/10.1109/TKDE.2013.39}.

\bibitem{nltkDoc} NLTK Project. 2023. NLTK 3.8 Documentation. NLTK Project website. \url{https://www.nltk.org/}.

\bibitem{pytorchDoc} PyTorch. 2023. PyTorch Documentation. PyTorch website. \url{https://pytorch.org/docs/}.

\bibitem{hfDoc} Hugging Face. 2023. Transformers Library Documentation. Hugging Face website. \url{https://huggingface.co/docs/transformers/}.

\bibitem{streamlitDoc} Streamlit. 2023. Streamlit Documentation. Streamlit website. \url{https://docs.streamlit.io/}.

\end{thebibliography}

\end{document}